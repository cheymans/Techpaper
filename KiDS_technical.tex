\documentclass[useAMS,usenatbib,times,letter,amssymb]{mn2e}
\usepackage{epsfig,times,amssymb,amsmath,color,verbatim}

\title[KiDS]{Technical Paper}
\author[K. Kuijken et al.]{Konrad Kuijken$^{1}$\thanks{kuijken@strw.leidenuniv.nl}, + KiDS Collaboration\\
%CFHTLenS Collaboration
\\
$^{1}$Leiden Observatory, Leiden University, Niels Bohrweg 2, 2333 CA Leiden, The Netherlands.\\
$^1$Scottish Universities Physics Alliance, Institute for Astronomy, University of Edinburgh, Royal Observatory, Blackford Hill, Edinburgh, EH9 3HJ, UK.\\ 
$^7$Mullard Space Science Laboratory, University College London, Holmbury St Mary, Dorking, Surrey RH5 6NT, UK.\\
$^8$Department of Physics and Astronomy, University of British Columbia, 6224 Agricultural Road, Vancouver, V6T 1Z1, BC, Canada.\\  
$^9$Argelander Institute for Astronomy, University of Bonn, Auf dem H{\"u}gel 71, 53121 Bonn, Germany.\\
$^{10}$Leiden Observatory, Leiden University, Niels Bohrweg 2, 2333 CA Leiden, The Netherlands.\\
$^{11}$Department of Physics and Astronomy, University of Victoria, Victoria, BC V8P 5C2, Canada.\\
$^{12}$Department of Physics, Oxford University, Keble Road, Oxford OX1 3RH, UK.\\ 
$^{20}$Department of Physics and Astronomy, University College London, Gower Street, London WC1E 6BT, UK.\\
}
\newcommand{\be}{\begin{equation}}  \newcommand{\ee}{\end{equation}}
\newcommand{\mb}[1]{\mbox{ #1 }}  
\newcommand{\ba}{\begin{eqnarray}}\newcommand{\ea}{\end{eqnarray}}
\newcommand{\bm}[1]{\mbox{\boldmath{$#1$}}}   %this is bold italic for MNRAS
%\newcommand{\bm}[1]{\bmath{$#1$}}

\renewcommand{\d}[0]{{\rm d}}
\newcommand{\ave}[1]{\langle #1 \rangle}
\newcommand{\Ave}[1]{\Big\langle #1 \Big\rangle}
\newcommand{\Ref}[1]{\ref{#1}}
\newcommand{\Cite}[1]{[\cite{#1}]}
\renewcommand{\vec}[1]{{\bmath{#1}}}
\newcommand{\mat}[1]{\mathbfss{#1}}

\newcommand{\red}[1]{{\color{red}{#1}}}

\newcommand{\kk}[1]{{\bf KK:}{\tt #1}}

\def\gs{\mathrel{\raise1.16pt\hbox{$>$}\kern-7.0pt %
\lower3.06pt\hbox{{$\scriptstyle \sim$}}}}         %
\def\ls{\mathrel{\raise1.16pt\hbox{$<$}\kern-7.0pt %
\lower3.06pt\hbox{{$\scriptstyle \sim$}}}}         %

%this offset does not work for me... KK
%\voffset=-0.6in


\begin{document}

\maketitle

\begin{abstract}

\end{abstract}


\begin{keywords}
cosmology: observations - gravitational lensing 
\end{keywords}

\section{Introduction}
\label{sec:intro}
\red{KK/JdJ} \\

Description of survey and Observations

\section{Data}
\red{KK/JdJ/TE} \\

Seeing distribution, optics etc, THELI reduction, masking, PSF patterns

\subsection{KiDS Photometry for Redshifts}
\red{KK,HH/AC}\\

%Gaap/gaussianisation, photozs
\subsubsection{Gaussian Aperture and PSF Photometry (GAAP)}

Photometric redshifts of galaxies require accurate colour measurements. These colours do not need to describe the total light from the galaxy: but they should represent the ratio of the flux from the same part of the galaxy in different filter bands. This means that, in order to optimize S/N, photometric redshifts can be based on the brighter, central regions of galaxies without the need to include the low surface brightness outskirts.
For small galaxies this requires correcting for the blurring effects of the point spread function.

One technical challenge is the fact that the point spread function is not constant: it varies from image to image, with position in each image, from sub-exposure to sub-exposure, and with wavelength. 
This variable PSF makes it difficult to measure useful aperture fluxes of galaxies: depending on the PSF, the flux measured inside a given aperture on the observed image will correspond to a different part of the galaxy itself.

We correct for the PSF variations in two steps: first by homogenizing the PSF within each coadded stacked image to a Gaussian shape without significantly degrading the seeing, and second by analytically correcting the aperture photometry of each galaxy for the seeing differences that remain. This approach has the advantage that the convolution to a homogenized PSF, which is the computationally more expensive step, only needs to be done once for each observation.

\subsubsection{PSF Gaussianization}

To homogenize the PSF we model the stars with shapelet expansions (\cite{Refregier2001}), and use these to construct and apply a suitable spatially varying convolution kernel. Shapelets are elementary, compact, orthonormal 2D functions which can be used to expand an image to arbitrary precision. They are characterized by a Gaussian scale radius $\beta$, multiplied by polynomial terms: the shapelet with Cartesian orders $(a,b)$ is (for $a,b=0,1,2,\ldots$)
\[
S_{ab}(x,y;\beta)={H_a(x/\beta)H_b(y/\beta) \over \beta\sqrt{2^{a+b}\sqrt\pi a! b!}}  e^{-(x^2+y^2)/2\beta^2} 
\]
where $H_a(x)$ is a Hermite polynomial, familiar from the eigenstates of the quantum harmonic oscillator, from which many of the useful properties of shapelets, such as their behaviour under infinitesimal translation, rotation, magnification, shear, and convolution may be derived (see \cite{Refregier2001} for details).  

Shapelets have been used to characterize the galaxy populations {\cite()}, to measure weak lensing shear {\cite()} and flexion {\cite()}, and ... {\cite()}.  
Even though the shapelets are formulated in cartesian coordinates, truncating the expansion at maximum combined order $N=a+b$ results in an orientation-invariant subspace of possible shapes that can be described. Shapelets are most sensitive to modelling structure at radii between $\beta/\sqrt{N+1}$ and $\beta\times\sqrt{N+1}$. At large radii they asymptotically approach a Gaussian.

Because of the orthonormality of the elementary shapelets, any source's image $I(x,y)$ can be described as a sum $\sum_{ab} s_{ab} S_{ab}(x,y;\beta)$ with coefficients
\[ 
s_{ab}=\int\,dx\,dy\, I(x,y) S_{ab}(x,y;\beta).
\] 
Since our data are pixellated we do not use this integral relation, but rather make a least-squares fit of a truncated shapelet model to the image of our source. We fit all pixels within a radius $(4+\sqrt{N})\beta$ of the center of the source.
 
Using this formalism, our image ``PSF gaussianization'' procedure is as follows:
\begin{enumerate}
\item
First we identify high-S/N unsaturated stars in the image, using the traditional flux vs. radius plot (\cite{KSB}). Typically several thousand such stars can be found per tile. Each star is then first fit with a simple Gaussian, and the median Gaussian radius $\sigma_{\rm psf}$ determined.  
Then all stars are modelled with a flux-normalized shapelet expansion to cartesian order 10, using a scale radius $\beta_{\rm psf}=1.3 \sigma_{\rm psf}$ chosen to prevent overfitting the cores of the stars while also reaching out into the wings of the PSF. All shapelet models use cartesian pixel coordinates $x,y$ with respect to a center position $(\xi,\eta)$ of the star: we define this center as the position for which $s_{10}=s_{01}=0$ and determine it iteratively.
\item
The PSF models are then interpolated across the image by means of (4th-order) polynomial fits of each shapelet coefficient versus $\xi$ and $\eta$ position (15 spatial variation coefficients per shapelet term). In this step outliers are rejected iteratively, resulting in a smooth model of the PSF variation across the image. The PSF model is thus described as a linear combination of $15\times66=990$ terms:
\[
P(x,y,\xi,\eta)=\sum_{a=0;b=0}^{a+b\le10} s_{ab} (\xi,\eta) S_{ab}(x,y;\beta_{\rm psf})
\]
where
\[
s_{ab}(\xi,\eta)=\sum_{k=0;l=0}^{k+l\le4} s_{ab;kl} (\xi-\xi_0)^k (\eta-\eta_0)^l 
\]
and $(\xi_0,\eta_0)$ is the center of the image.
\kk{Show some illustrative models. Discuss lack of PSF jumps?}
\item
Having constructed a map of the PSF, the next step is to construct a convolution kernel that renders the PSF Gaussian. Also here the shapelet formalism is very convenient, since shapelets behave nicely under convolution (\cite{Regfregier2011}). 

construct kernel
\item 
noise correlation function
\item
convolution
\item
show resulting PSFs
\end{enumerate}

Note that even though, for practical reasons, we carry out the photometry as a two-step process (first manipulating the pixels in the image and then photometering the result) it can also be written as a single photometry step on the original variable-PSF stack (albeit with a complicated aperture function). Effectively, therefore, our photometry is still a linear combination of calibrated pixel fluxes, with a tractable error analysis. 



\subsubsection{Gaap photometry}

\kk{TBW}

formalism

choice of aperture size and orientation

examples of S/N optimization in practice



mention stack-before-gauss vs gauss-b4-stack -- for now do the former (cheaper). Can we show some comparisons?


other uses of gaussianized images:

cleaner star/gal separation (refer to Pila et al)

shear measurements?

galaxy morphology? colour gradients? 


\red{Hendrik: photz's}

\subsection{KiDS Shapes for Lensing}
\red{CH/LM/RN}\\

PSF modelling, noise biases, Lensfit updates '\\


\subsection{KiDS tests for systematic errors}
\red{CH/CB/KK/AC}

PSF small-scale structure? (MV). Look at denser KiDS fields and compare subsampled PSF maps to see whether sparser fields have sufficient information to model PSF. Also can look at small-scale correlation of PSF ellipticities. Or 47Tuc data.

Systematics tests: star-gal and g-g lensing (zl,zs), photo-z angular correlatio between z-bins

\section{Weak Lensing by large-scale structure}
\red{KK/CH} \\
Some theory\\
Cosmic shear in pass/fail fields/ regions of high/low psfs\\
Comparison to CFHTLenS\\
KK idea:
\begin{verbatim}

DO THIS WITH GG LENSING FIRST?

And had another idea for something to put in the technical paper, see what you think:

 even if we cannot have cosmic shear analyses for early science, we can I think present
 some correlation functions and compare them to cfhtlens. I think this would be interesting
 from a technical point of view, as follows:
- cfhtlens is very interesting, but there are question marks about the calibration
- kids covers a similar area, but to less sensitivity
- lensfit needs s/n dependent calibration.
As a test of our s/n-dependent calibration we can make corr.fns (maybe even just xi+) of 
galaxies in r-band magnitude bins, for KiDS and CFHTLens (and RCS2?). If the calibrations 
are OK these should agree up to the errors. This is a meaningful test because the same 
magnitude bin is observed with different S/N in the different surveys. Agreement validates 
(even if it does not prove) the S/N correction. Disagreement should be a red flag, or at 
least a warning.
I like this test because 
- if it works it shows the S/N calibration is consistent across surveys
- it is cosmology independent 
- it does not require photo-z
- the measurements should be relatively straightforward.
Of course the actual comparison can only be made once we unblind.
\end{verbatim}


\section{Conclusions}
\red{KK}\\
KiDS is great hooray!






\section{Acknowledgements}

{\small Author Contributions: All authors contributed to the development and writing of this paper etc etc}

\bibliographystyle{mn2e}
\bibliography{ceh_2012}
\label{lastpage}


\end{document}

