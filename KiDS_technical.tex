\documentclass[useAMS,usenatbib,times,letter,amssymb]{mn2e}
\usepackage{epsfig,times,amssymb,amsmath,color,verbatim}

\title[KiDS]{Technical Paper}
\author[K. Kuijken et al.]{Koen Kuijken$^{1}$\thanks{kuijken@strw.leidenuniv.nl}, + KiDS Collaboration\\
%CFHTLenS Collaboration
\\
$^{1}$Leiden Observatory, Leiden University, Niels Bohrweg 2, 2333 CA Leiden, The Netherlands.\\
$^1$Scottish Universities Physics Alliance, Institute for Astronomy, University of Edinburgh, Royal Observatory, Blackford Hill, Edinburgh, EH9 3HJ, UK.\\ 
$^7$Mullard Space Science Laboratory, University College London, Holmbury St Mary, Dorking, Surrey RH5 6NT, UK.\\
$^8$Department of Physics and Astronomy, University of British Columbia, 6224 Agricultural Road, Vancouver, V6T 1Z1, BC, Canada.\\  
$^9$Argelander Institute for Astronomy, University of Bonn, Auf dem H{\"u}gel 71, 53121 Bonn, Germany.\\
$^{10}$Leiden Observatory, Leiden University, Niels Bohrweg 2, 2333 CA Leiden, The Netherlands.\\
$^{11}$Department of Physics and Astronomy, University of Victoria, Victoria, BC V8P 5C2, Canada.\\
$^{12}$Department of Physics, Oxford University, Keble Road, Oxford OX1 3RH, UK.\\ 
$^{20}$Department of Physics and Astronomy, University College London, Gower Street, London WC1E 6BT, UK.\\
}
\newcommand{\be}{\begin{equation}}  \newcommand{\ee}{\end{equation}}
\newcommand{\mb}[1]{\mbox{ #1 }}  
\newcommand{\ba}{\begin{eqnarray}}\newcommand{\ea}{\end{eqnarray}}
\newcommand{\bm}[1]{\mbox{\boldmath{$#1$}}}   %this is bold italic for MNRAS
%\newcommand{\bm}[1]{\bmath{$#1$}}

\renewcommand{\d}[0]{{\rm d}}
\newcommand{\ave}[1]{\langle #1 \rangle}
\newcommand{\Ave}[1]{\Big\langle #1 \Big\rangle}
\newcommand{\Ref}[1]{\ref{#1}}
\newcommand{\Cite}[1]{[\cite{#1}]}
\renewcommand{\vec}[1]{{\bmath{#1}}}
\newcommand{\mat}[1]{\mathbfss{#1}}

\newcommand{\red}[1]{{\color{red}{#1}}}


\def\gs{\mathrel{\raise1.16pt\hbox{$>$}\kern-7.0pt %
\lower3.06pt\hbox{{$\scriptstyle \sim$}}}}         %
\def\ls{\mathrel{\raise1.16pt\hbox{$<$}\kern-7.0pt %
\lower3.06pt\hbox{{$\scriptstyle \sim$}}}}         %

\voffset=-0.6in


\begin{document}

\maketitle

\begin{abstract}

\end{abstract}


\begin{keywords}
cosmology: observations - gravitational lensing 
\end{keywords}

\section{Introduction}
\label{sec:intro}
\red{KK/JdJ} \\

Description of survey and Observations

\section{Data}
\red{KK/JdJ/TE} \\

Seeing distribution, optics etc, THELI reduction, masking, PSF patterns

\subsection{KiDS Photometry for Redshifts}
\red{KK,HH/AC}\\

Gaap/gaussianisation, photozs

\subsection{KiDS Shapes for Lensing}
\red{CH/LM/RN}\\

PSF modelling, noise biases, Lensfit updates '\\


\subsection{KiDS tests for systematic errors}
\red{CH/CB/KK/AC}

Systematics tests: star-gal and g-g lensing (zl,zs), photo-z angular correlatio between z-bins

\section{Weak Lensing by large-scale structure}
\red{KK/CH} \\
Some theory\\
Cosmic shear in pass/fail fields/ regions of high/low psfs\\
Comparison to CFHTLenS\\
KK idea

\section{Conclusions}
\red{KK}\\
KiDS is great hooray!






\section{Acknowledgements}

{\small Author Contributions: All authors contributed to the development and writing of this paper etc etc}

\bibliographystyle{mn2e}
\bibliography{ceh_2012}
\label{lastpage}


\end{document}

